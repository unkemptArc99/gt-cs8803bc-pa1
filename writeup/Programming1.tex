
\documentclass[11pt]{article}
\usepackage[margin = 1in]{geometry}
%
% ADD PACKAGES here:
%

\usepackage{amsmath,amsfonts,graphicx,amssymb}
\usepackage{tikz}
\usetikzlibrary{arrows}
\usepackage{braket}
\usepackage{cite}
\usepackage{changepage}
\usepackage{hyperref}
\usepackage{mathtools}

\usetikzlibrary{arrows.meta}
%
% The following commands set up the lecnum (lecture number)
% counter and make various numbering schemes work relative
% to the lecture number.
%
\newcounter{lecnum}
\renewcommand{\thepage}{\thelecnum-\arabic{page}}
\renewcommand{\thesection}{\thelecnum.\arabic{section}}
\renewcommand{\theequation}{\thelecnum.\arabic{equation}}
\renewcommand{\thefigure}{\thelecnum.\arabic{figure}}
\renewcommand{\thetable}{\thelecnum.\arabic{table}}

%
% The following macro is used to generate the header.
%
\newcommand{\lecture}[4]{
   \pagestyle{myheadings}
   \thispagestyle{plain}
   \newpage
   \setcounter{lecnum}{#1}
   \setcounter{page}{1}
   \noindent
   \begin{center}
   \framebox{
      \vbox{\vspace{2mm}
    \hbox to 6.28in { {\bf CS 8803/4803 Blockchain and Cryptocurrencies.
		\hfill } }
       \vspace{4mm}
       \hbox to 6.28in { {\Large \hfill Solidity Programming Assignment 1  \hfill} }
       \vspace{2mm}
       \hbox to 6.28in { {\it Name: #3 \hfill} }
      \vspace{2mm}}
   }
   \end{center}
   \markboth{Lecture #1: #2}{Lecture #1: #2}

   \vspace*{4mm}
}
%
% Convention for citations is authors' initials followed by the year.
% For example, to cite a paper by Leighton and Maggs you would type
% \cite{LM89}, and to cite a paper by Strassen you would type \cite{S69}.
% (To avoid bibliography problems, for now we redefine the \cite command.)
% Also commands that create a suitable format for the reference list.
\renewcommand{\cite}[1]{[#1]}
\def\beginrefs{\begin{list}%
        {[\arabic{equation}]}{\usecounter{equation}
         \setlength{\leftmargin}{2.0truecm}\setlength{\labelsep}{0.4truecm}%
         \setlength{\labelwidth}{1.6truecm}}}
\def\endrefs{\end{list}}
\def\bibentry#1{\item[\hbox{[#1]}]}

%Use this command for a figure; it puts a figure in wherever you want it.
%usage: \fig{NUMBER}{SPACE-IN-INCHES}{CAPTION}
\newcommand{\fig}[3]{
			\vspace{#2}
			\begin{center}
			Figure \thelecnum.#1:~#3
			\end{center}
	}
% Use these for theorems, lemmas, proofs, etc.
\newtheorem{theorem}{Theorem}[lecnum]
\newtheorem{lemma}[theorem]{Lemma}
\newtheorem{proposition}[theorem]{Proposition}
\newtheorem{claim}[theorem]{Claim}
\newtheorem{corollary}[theorem]{Corollary}
\newtheorem{definition}[theorem]{Definition}
\newenvironment{proof}{{\bf Proof:}}{\hfill\rule{2mm}{2mm}}

% **** IF YOU WANT TO DEFINE ADDITIONAL MACROS FOR YOURSELF, PUT THEM HERE:

\newcommand\E{\mathbb{E}}

\begin{document}
%FILL IN THE RIGHT INFO.
%\lecture{**LECTURE-NUMBER**}{**DATE**}{**LECTURER**}{**SCRIBE**}
\lecture{1}{August 24}{Abrahim Ladha, Stan Peceny}{scribe-name}
%\footnotetext{These notes are partially based on those of Nigel Mansell.}

\section{Part 0, Setting up Truffle (0 Points)}

Please see the attached \texttt{readme.txt} for instructions on how to set up Truffle. We highly recommend that you go over the \href{https://cryptozombies.io/}{CryptoZombies Tutorial}.

\section{Part 1 (80 Points)} 
\noindent
You are provided a skeleton contract, and you are to implement the empty functions it contains. \\

\noindent
The attached \texttt{readme.txt} contains detailed instructions and examples. \\

\noindent
We will run 10 test cases, each is worth 8 points (5 test cases for iterative bubble sort, 5 for recursive).

\section{Part 2 (20 Points)}
Answer the following questions, please provide formal and precise explanations. \\

\noindent
Each question is worth 2 points but question 5, which is worth 6 points. \\
\noindent
Bonus question is worth 2 points.\\


Solidity
\begin{enumerate}
	\item What version of the solidity compiler are you using with truffle? How did you determine this using truffle?
	\item Explain the parameters for the first line. \texttt{pragma solidity}. What do each of the $\wedge$,$<=,<$ operators do?
	
	
	\item What is the maximum size of \texttt{uint} type in solidity? Give your answer in terms of bits. Why do you think this size is the default for \texttt{uint} in Solidity?
	
	
	\item What is the difference between a \texttt{pure} and a \texttt{view} function in Solidity? Please provide and explain a use case for a \texttt{pure} function and a use case for a \texttt{view} function.
	
	\item How much gas was used to deploy (not execute) the 1. iterative bubble sort function and the 2. recursive bubble sort function of your contract? Explain how you were able to compute this. Compare the gas used between each function and give a reason which would explain the difference. 
	
	\item Is the gas used to deploy deterministic? Explain.
	\item How many local variables can you have in a Solidity function?


	\item What is \texttt{msg.sender}? 
	
	
	\item[(Bonus)] Is Solidity Turing Complete? Please provide an in depth answer. There are a lot of wrong answers for both yes and no online, so be careful what you read. 

\end{enumerate}


Your deliverables include only two files. \texttt{Sorter\_YOUR\_NAME.sol} which is the skeleton contract provided, but fully implemented, and \texttt{Report\_YOUR\_NAME.pdf}, which is your solution to the questions in part 2. Please do not zip the files. 
\end{document}
 
